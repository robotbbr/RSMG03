\chapter{Research Problem}

\section{Human activity analysis}
% Description of the problem

%TODO Main sentence for the problem

The application domain will dictate the activities to be recognized, the required precision, the sensors to be used and their location (environment located or wearable), the time constraints for deliberation,



%The main goal is to automatically analyse the ongoing activities from a sensory source (a video sequence in most of the cases).

%Activities can be understood in physical terms (space and time), but also symbolically as they can usually be associated with a meaning and a context. 

%Human activities are difficult to classify because they cover a broad range of situations and they depend on different parameters.
%Regarding their complexity, activities can be treated as hierarchical entities because high-level activities are usually composed of simpler actions.
Regarding the complexity of the activities, in \citep{Turaga2008_MaRecHuAcSurv}, two non-exclusive categories are used: actions and activities.
The first one is used for simple actions performed preferably by a individual, and activities are treated as a complex sequence of actions performed by several individuals. 
On a similar fashion, in \citep{Aggarwal11_HumanActivity}, a four layers categorization is proposed:

\begin{description}
\item[Gestures] Elementary movements of a person's body part, and are the atomic components describing a meaningful motion of a person. 
E.g. `stretching an arm', `raising a leg'.
\item[Actions] Single person activities that may be composed of multiple gestures organized temporally. 
E.g. `walking', `waving'.
\item[Interacions] Human activities that involve two or more persons and/or objects. 
E.g. `Two persons fighting', `a person eating an apple'.
\item[Group activities] The activities performed by conceptual groups of multiple persons and/or objects. 
E.g. `a football team playing a match', `a group of students making an exam'.
\end{description}









% Alternatives for each part
% > Observations
% > Knowledge (where? & how?, I'll do it by hand, but it's important to mention alternatives).
% > Representations
% *** CHECK THE orgnanization from the surveys
%   - Modelling activities in space (QSR)
%   - Modelling activities in time  (QSTR ~ Allen's, Fluent, Event, etc).
%   - Modelling activities semantically (Ontologies + Hierarchies)
% > Inference

% Test example

\section{Proposal} % Or alternatives

% Main parts of the problem

% A - Scene decomposition (locations, objects, persons)
%   > from observations to scene reconstruction

% B - 

% B - Representation 
% Modelling in space (QSR)
% Modelling in time (QSTR)
