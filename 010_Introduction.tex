\chapter{Introduction}

% INTRO 
% > Robots in everyday environments are important.
% > This environment is hard.
% > Robots needs cognitive skills: understand the environment.
One of the main goals in AI is having robots working autonomously in everyday environments. A robot in this kind of situation is expected to perceive, understand and interact with his environment. However, the environment is dynamic, non-structured and non-deterministic, which makes difficult for a robot to fulfil assigned tasks. To be able to sort these obstacles, robots need to be provided with cognitive skills, in particular the capacity to perceive the environment to interpret and understand it. %glsdef_{cognition}

% GENERALITIES ABOUT HUMAN ACTIVITIES
% > Activities are important part of the scenes and they are necessary to be able to understand the scene
% > Activities are entities with a spatio-temporal & symbolic (semantic, hierarchical) characteristics
Everyday environments have many valuable features that a robot needs to understand, among them are human activities. They are a manifestation of human behaviour that can be associated to a concept. They are important for a robot in order to be able to understand the role of humans and the interactions with objects and the environment in a particular scene.

%\section{Research Problem}
\section{Human activity analysis with a mobile robot}
% ACTIVITY RECOGNITION
% > AR as a field
% > Robot case, advantages and disadvantages of using a robot
Activity recognition is the research field in charge of the automatic analysis of human activities. The problem has been treated from different perspectives, principally by analysing video sequences, but also by using a multi-layered and hierarchical approach. The area is still active and the use of a robot has not been fully exploited.

Robots are capable of performing activity recognition and they have some advantages over fixed cameras. They are active observers, i.e. they can change the point of view and be selective on the areas in the environment that are more interesting to observe. On the other hand, they have some disadvantages as well. They don't have omnipresence so they will loose information. They also may have erratic data from the sensors due to movement, different environmental conditions, etc. The robots are also expected to work in real-time, so a quick deliberation would be desirable.

% RESEARCH PROBLEM
% > Talk about AR with a robot in cases with incomplete information
% > Generalities of the problem 
% > Example case of a sequence
Due to their nature, robots are susceptible to handle erratic or incomplete data. However, it is important to explore alternatives to tackle these problems due to the potential advantages of robots. This is the motivation for this project, the study of activity recognition performed with a mobile robot and focusing in the cases where complete information from the environment is not available.

% Maybe discuss about applications.
% Quick draft of the proposal.
% Still need to discuss advantages and disadvantages of the problem, and of the proposed "solution".

%\section{Approaches}
% 


