\chapter{Introduction}

% INTRO 
% > Robots in everyday environments are important.
% > This environment is hard.
% > Robots needs cognitive skills: understand the environment.
One of the main goals in AI is having robots working autonomously in everyday environments. 
In such situation, robots are expected to perceive, understand and interact with its environment. 
However, these kind of environments are dynamic, non-structured and non-deterministic, which makes difficult for a robot to fulfil the assigned tasks. 
To be able to sort these obstacles, robots need to be provided with cognitive skills.

Human cognition refers to all mental activities associated with thinking, knowing, remembering and communicating, and how the information is processed \citep{King2014Psychology,myers2013psychology}. 
In robotics, the concept is associated with systems that emulate these mental processes or those that \textit{sense}, \textit{plan} and \textit{act}.
More precisely, cognition can refer to those systems that can perceive, understand \ldots and interact with their environment, and evolve in order to achieve human-like performance in activities requiring context (situation and task) specific knowledge \citep{christensen2010cognitive}.

% GENERALITIES ABOUT HUMAN ACTIVITIES
% > Activities are important part of the scenes and they are necessary to be able to understand the scene
% > Activities are entities with a spatio-temporal & symbolic (semantic, hierarchical) characteristics
Everyday environments have many valuable features that a robot needs to understand, in order to succeed while performing a task, among them are human activities.
Human activities are a meaningful manifestation of human behaviour along time and space. %TODO Expand the definition of activity.
They provide rich information about the human performing it, but also, about his/her relation with other relevant components of the environment as humans, objects and locations.


%\section{Research Problem}
\section{Human activity analysis with a mobile robot}
% ACTIVITY RECOGNITION
% > AR as a field
% > Robot case, advantages and disadvantages of using a robot
Activity recognition is the research field that studies the automatic detection and analysis of human activities by processing the data acquired from sensors \citep{Aggarwal14_HumActRec3DRev}. 
It is not restricted only to sensory data but it can also be complemented with other sources of information, i.e. domain knowledge.
In the AI context, activity recognition is closely related with the areas of perception, knowledge representation and reasoning. 
The problem of activity recognition has been treated from different perspectives, however, computer vision has been the most popular.

% Robots in AR (general)
In principle, robots with appropriate sensing and processing capabilities can perform activity recognition. 
Moreover, they have some advantages over the use of fixed cameras or wearable devices as they are able to interact with the environment. 
Robots are active observers, i.e. they can change their point of view on scene and be selective in the areas of the environment that are more interesting. 
On the other hand, they have some disadvantages as well. 
They don't have omnipresence, so they are not able to sense the full environment and will loose information.
Also, their sensors have constraints, the data may be noisy or blurry due to movement, erratic hardware, changing environmental conditions, etc. 
Finally, robots are expected to work in real-time, so online activity recognition is desirable but yet difficult to achieve.

% BRIDGE TO ASP
Activities involve knowledge.
They associate concepts and relations between a subject and the environment.
In general, an activity recognition system should be able to handle, not only sensory data but also symbolic representations, and be able link top level symbolic concepts to low level sensory data; this is known as the \textit{anchoring problem} \citep{Coradeschi03_AnchoringProblem}.
With this in mind, knowledge processing and reasoning is a necessity for such systems.


\section{Answer Set Programming for Knowledge Representation and Reasoning}

There have been proposed many ideas to handle the problems of knowledge representation and reasoning (e.g. logic programming, ontologies, bayesian networks, fuzzy logic, etc.), among them is Answer Set Programming (ASP).
ASP is form of declarative programming oriented towards difficult, primary NP-hard, search problems.
It establish a new paradigm of logic programming that allows concepts as negation as a failure, default knowledge and non-monotonic reasoning.

ASP main concepts were proposed since the late 1980s \citep{Gelf88a} and it has been used with success in many applications.
Traditionally, ASP solvers were designed as one-shot problem solvers, so they lacked of reactive capabilities and, for example, whenever new data arrived, the system needed to be restarted.
This has been one of the main reasons why ASP has not been fully exploited in the field of robotics, however, in recent years an important effort has been directed towards this direction by some groups \citep{AndresOSSR13_rosoclingo,Erdem2013_IntLowRTaskP}.


\section{Research Problem}

This project is based in the consideration of ASP as a solid option for robots in problems that require symbolic representations.
The goal of this work is to study the integration of ASP in a robotic system as a tool for knowledge representation and reasoning in the context of the problem of activity recognition.

ASP allows the manipulation of incomplete information and handling multiple sources of knowledge.
The integration between observations and external knowledge appear to be a more robust approach than a single sensory based approach.
As mentioned before, hardware adds uncertainty, is constrained and is usually difficult to process; so putting emphasis in an upper layer of knowledge representation and reasoning would enable, not to eradicate, but to complement a sensory based approach and also to minimize its charge in a system.

% RESEARCH PROBLEM
% > Talk about AR with a robot in cases with incomplete information
% > Generalities of the problem 
% > Example case of a sequence
%TODO %TODO %TODO %TODO %TODO %TODO %TODO %TODO %TODO %TODO %TODO %TODO %TODO %TODO %TODO %TODO

%Persentation of the project
%Particularly in the case where there is not complete information from the environment to have a clear match between the observations and %the activity patterns. 
%Here, an interpretation can still be made using previous experience and domain knowledge. 
%Even, if a totally certain interpretation of the scene is not possible, a partial one can still be done with a list of the most probable situations to be happening. 
%This also can be used by a robot to decide to perform new observations of the scene to improve its reasoning conclusions. 
%The chosen technique to do this is Answer Set Programming (ASP).

%TODO Write paragraph about ASP
%Answer Set Programming is a logical p

% Quick draft of the proposal.

\section{Test scenario: ``The library setting''} \label{sec_Library} %TODO Add images.

The School of Computer Science at the University of Birmingham has a library, mostly used by students.
An attendant is in charge of the book loans and retrievals, and also to assist the users.
The physical scenario is basically a big room. It has some cabinets (where bibliographic material is stored), a reception, some tables with chairs and a printing desk. It only has one entrance.

Users mostly use the facility to study, to consult material, to print, to work in team, to do work in PC or simply as a rest area.
Because of the rules of the library and nature of the scenario, the amount of activities is restricted by the domain.
However, some other activities could appear as using a cellphone, talking, putting things inside a backpack, etc.
The objects involved in the scenario is relatively small (books, tables, chairs, laptops, cabinets, etc.).

There are many factors that from which will depend the components and the complexity of an activity recognition system, the targeted activities is an important one.
Within the scope of this project, a library offers an ideal stage to test an activity recognition system.
It provides a challenging and interesting environment, while still maintain the amount of objects and activities limited.
This last one is important, because it maintains \textit{bounded} the amount of domain knowledge that will be required in a library, compared with other more broad environments, in term of activities.


%Libraries in general are very similar, although they may vary in size (e.g. buildings or simple rooms), in target users (e.g. graduate students, children) or in other aspects. However, there are some activities that are 





