\documentclass[a4paper,12pt,openany,oneside]{report}
\usepackage{url}

\begin{document}

The problem to be studied in this research project is activity recognition with a mobile robot using Answer Set Programming. We are interested in the spatio-temporal relation between human activities and the environment, and how this knowledge can be used by a mobile robot. 

Activity recognition is the research field that studies the automatic detection and analysis of human activities by processing the data acquired from sensors. Answer Set Programming (ASP) is a declarative problem solving technique initially conceived for NP-hard search problems, but has also been used for knowledge representation and reasoning. ASP allows non-monotonic reasoning, negation as a failure and constraint programming. Finally, robots are a desirable platform to run activity recognition as they usually work in human environment. Robots have some advantages, particularly the possibility to participate and interact with the work, this enables an active perception approach which stands ahead of static point of view approaches.

ASP is used for knowledge representation and reasoning. Activities and domain knowledge needs to be available via ASP to process incoming observations from the world and use them to create a conclusion of the current state of the world. This knowledge is useful for a robot to create a reliable internal model of the world, bor a better understanding of the environment and to generate action plans to react to situations. 

As human activities are diverse, the current work will be centred on recognizing the activities of a specific domain: a library. An experimental approach is proposed that goes through implementing a static simulation, a 3D simulation, handling real data and eventually running tests with a mobile robot in a real library. 

In the end, the interest relies in provide robots with the capacity to perceive human activities, handle them via ASP and semantic maps of activities and, finally, to use the resulting knowledge in current or future action plans.






%\clearpage
%\bibliographystyle{plain}
%\bibliography{SchedulerSummary}
\end{document}