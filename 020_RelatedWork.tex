\chapter{Related Work}

% 0 - GENERALITIES
% Perception
% Symbol ground
% Anchoring
% Frame Problem

Human activity recognition is an important research area in the context of automated perception. It has many applications as surveillance, inspection, verification, etc. The problem has been treated from different perspectives,  andverification, in aabout the problem of interpretation and understanding of a scene from sensory information and also using domain knowledge. 
It is mostly a perceptual action, however reasoning plays an important role too.

% Check the surveys to get the whole image



% 1 - ACTIVITY RECOGNITION PROBLEM
% Historic origins
% Approaches (camera, wearable device, ubiquitous computing)
% Main branches (single layer, multpiple layer)
% Hierarchical approaches

% 1.5 ACTIVITY RECOGNITION WITH A MOBILE ROBOT
A robot can be roughly conceived as a physical entity capable of sensing and performing actions in the world.
With this in mind, it seems clear that a robot, with sufficient sensing capabilities, is a good candidate to perform the task.


% 2 - ANSWER SET PROGRAMMING
% General background
% Related work to AR


% 3 - ASP + Robots -> ROSoClingo
% 
