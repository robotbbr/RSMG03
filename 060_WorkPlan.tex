\chapter{Work Plan}

The main goal of this project is to present a concise study of the integration of ASP and robotics to treat the problem of activity recognition.
By this mean, a work plan should be submitted as a guide line to direct future activities and to measure the progress in the the project. In the following sections, two plans are are presented, short-term (6 months) and long-term (2 years). The proposed goals are distributed in three groups:

First \textit{project goals} are those which moves forward the research project. 
As the direction of research is now clear (ASP-based activity recognition with a mobile robot), and the literature review is considerably up to date, these goals are in the sense of the implementation, testing and analysis of the proposed methodology, to find the paths that conduce to better results. 
At the beginning these tasks will be simple and look to proof the feasibility of the proposed ideas. 
In the final stages, the tasks will be driving the project towards more realistic cases and/or towards specific problems that proof to be relevant.

\textit{Research goals} are those which serve for research training purposes and those enable the participation of the current project in activities within the research community, e.g. conferences, research school meetings, presentations, academic writing, etc.

Finally, the \textit{academic responsibilities} are conformed with those tasks that are mandatory for PhD students within an academic program.


\section{Long-term Plan}

%In the long-term, the main goal is the proper completion of the project by the fulfilment of minor goals. 

\subsection{Project Goals}

\subsubsection*{T1 - Library Setting 01: Terminal simulation}
\paragraph{Objectives}
\begin{itemize}
\item Start the implementation of an activity recognition system using the 2D library setting.
\item Explore simple representations of activities.
\item Get familiar with ASP tools (Clingo, ROSoClingo).
\end{itemize}
\paragraph{Tasks}
\begin{enumerate}
\item The implementation of a simple (terminal) activity recognition system using ASP tools (Potassco).
\item Test different representations of activities based on the interaction between spaces (regions) and humans (positions).
\item Test sequential representation of activities, i.e. a human visiting different locations.
\item Consider objects in scene and within the activity representations.
%\item \textit{Optional} Add visualisation software, e.g. NetLogo.
%\item \textit{Optional} Consider simple robot activity.
\end{enumerate}


\subsubsection*{T2 - Library Setting 02: MORSE Simulation}
\paragraph{Objectives}
\begin{itemize}
\item Implement a more realistic 3D simulation of the library setting.
\item Integrate ROS, Potassco and MORSE in a system.
\item Extend the representation of activities to a 3D environment; i.e. start considering geometrical entities as regions, trajectories, etc.
\end{itemize}
\paragraph{Tasks}
\begin{enumerate}
\item Create the library setting in MORSE and attack one robot to the environment.
\item Integrate ROS and Potassco (via ROSoClingo) with the MORSE simulation.
\item Explore spatial
\end{enumerate}


\subsection{Research Goals}

\subsection{Academic Goals}







\section{Short-term plan - 6 months}

In the short term, the goals will be focused in an introductory p, implementation and experimentation of the proposed methodology.
Proper mastery of ASP is desired to make a proper use of its capabilities.

The work will be oriented with two purposes.

\begin{itemize}
\item Get a deeper understanding of ASP, and particularly in the context of robotics.
\item Board the problem of activity recognition with simple cases.
\end{itemize}



The first point is mentioned because ASP has a central role in this project, is important to complete a proper analysis from insight about the possibilities and requirements to achieve the goal.
This includes the possible synergy with ASP and other techniques (e.g. probabilistic approaches) to improve the possibilities of an overall system.

Regarding the second point, activity recognition is the problem to treated.
A simple solution should be built, in order to have a platform for testing and also build more complex implementations.

\subsection{Short-term table}

\begin{landscape}

\begin{ganttchart}[hgrid,
%vgrid={*1{blue, ultra thick}, dotted},
vgrid,
x unit=7mm,time slot format=isodate, compress calendar]{2015-04-02}{2017-06-31}
\gantttitlecalendar{year, month} \\
%\gantttitlecalendar{year, month } \\
%\ganttgroup{Absences}{2015-04-02}{2015-04-06} \\
\ganttbar{}{2015-04-06}{2015-01-31}
\end{ganttchart}



%\ganttset{%
%calendar week text={W~\currentweek}
%}
%\begin{ganttchart}[hgrid,vgrid,x unit=4mm,time slot format=isodate]{2015-04-02}{2015-10-02}
%\gantttitlecalendar{year, month=shortname, week} \\
%%\gantttitlecalendar{year, month } \\
%\ganttbar{}{2015-04-06}{2015-07-02}
%\end{ganttchart}



%\begin{ganttchart}{1}{12}
%\gantttitle{2015}{12} \\
%\gantttitlelist{1,...,12}{1} \\
%\ganttgroup{Group 1}{1}{7} \\
%\ganttbar{Task 1}{1}{2} \\
%%\ganttlinkedbar{Task 2}{3}{7} \ganttnewline
%%\ganttmilestone{Milestone}{7} \ganttnewline
%%\ganttbar{Final Task}{8}{12}
%%\ganttlink{elem2}{elem3}
%%\ganttlink{elem3}{elem4}
%\end{ganttchart}




%\begin{ganttchart}{1}{12}
%\gantttitle{2015}{12} \\
%\gantttitlelist{1,...,12}{1} \\
%\ganttgroup{Group 1}{1}{7} \\
%\ganttbar{Task 1}{1}{2} \\
%\ganttlinkedbar{Task 2}{3}{7} \ganttnewline
%\ganttmilestone{Milestone}{7} \ganttnewline
%\ganttbar{Final Task}{8}{12}
%\ganttlink{elem2}{elem3}
%\ganttlink{elem3}{elem4}
%\end{ganttchart}




%\newpage

\end{landscape}




\begin{verbatim}
1 - Mini-World Simulation

2 - Library 3D Simulation

3 - Library Datasets (actors & controlled environment)

4 - Library (real environment)
\end{verbatim}


%\begin{itemize}
%\item Experiments 1 - Proof of concept
% \begin{itemize}
% \item Simulation - Simple ASP console example
% \item Simulation - Morse
% \end{itemize}
%\item Experiments 2 - Improve first experimental phase
% \begin{itemize}
% \item Expand activity representation and repertoire to handle more complicated cases
% \item Implement approach
% \end{itemize}
%\end{itemize}






