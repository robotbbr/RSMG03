\chapter{Conclusions} \label{ch_conc}

This report has presented a research project proposal about activity recognition with a mobile robot using ASP.

The problem of activity recognition is difficult and extensive. 
It can be treated with different approaches. 
Even more, activities, as languages, are evolving and cultural entities. 
However, the proper understanding of activities is an important skill to understand human environments.
This justifies the studies of the problem and the goal to provide machines with activity recognition capabilities.

This project makes emphasis in the use of a robot, which has some evident limitations.
Sensors are constrained to defined ranges and the thrown data usually presents noise.
Actuators also present inaccuracies and time-delayed responses.
Processing capabilities are also limited in a robot, so the amount of data collected with a robot should be minimized and processed efficiently whenever is possible.
This is, however, the current state in robotics and this is considered by considering sensor models, 

ASP is considered here as an interesting alternative for knowledge representation and reasoning purposes that has not been fully exploited in the context of robotics and activity recognition.
ASP provides robust and well supported tools for declarative problem solving.
However, it should also be used with reserve as it has been conceived to model combinatorial NP-hard problems, so the complexity of large systems can be a problem, but not in this project as the targeted environment for testing (library setting) is limited.

The experimental approach presented in this report goes from simple cases towards more realistic ones. 
The methodology will be evaluated by self and external comparison, and targeted running systems with a robotic platform.

Finally, as work plan has been proposed with specific goals and tasks to achieve it. 
Research and academic goals are considered.
